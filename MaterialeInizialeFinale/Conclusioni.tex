% !TEX encoding = UTF-8
% !TEX TS-program = pdflatex
% !TEX root = ../Tesi.tex
% !TEX spellcheck = it-IT

%*******************************************************
% Introduzione
%*******************************************************
In questo lavoro di tesi è stata trattata la problematica della sicurezza, evidenziando l'importanza che essa riveste dal punto di vista sociale e le difficoltà sociali e tecniche incontrate nella realizzazione di alcuni progetti.
La trattazione effettuata, incentrata sull'estrazione di entità nominali da documenti testuali e sulla predizione ed evoluzione delle categorie criminali nel tempo, non pretende di essere esaustiva, ma piuttosto un punto di partenza per ulteriori sviluppi, sia teorici che sperimentali. 

Il sistema TB-CREDIS è stato esteso per permettere l'estrazione dei perpetratori di un crimine all'interno di documenti più o meno segretati, che possono spaziare da articoli giornalistici a report investigativi, e predire le evoluzioni dei criminali, relativamente alle loro attività, in quegli intervalli temporali per cui non si possiede alcun documento associato al soggetto in esame relativo a tal periodo. L'alta modularità del sistema ha permesso inoltre la facile integrazione della componente di interfaccia utente: l' utente può caricare un nuovo criminale nella collezione di dati, effettuare le operazioni di predizione e visualizzare graficamente i risultati prodotti e le evoluzioni del criminale sotto analisi nel tempo.

I risultati sperimentali prodotti si sono rivelati all'altezza delle aspettative e incentivano a proseguire gli studi in questa direzione in modo da individuare nuove tecniche che permettano di migliorare i risultati raggiunti in termini di tempi di elaborazione e di qualità. 
In particolare la componente di estrazione delle entità perpetratrici può essere estesa con nuove euristiche per permettere di estrarre le entità nominali da quei pattern ad oggi non considerati, o estrarre nuove entità quali nomi di possibili vittime, luoghi, proprietà private e tutto ciò che possa rivelarsi utile alle attività investigative. 
Per la componente di predizione possibili sviluppi futuri potrebbero riguardate l'analisi e individuazione dei documenti relativi allo stesso reato, commesso da un criminale, in modo da evitare che reati più ``documentati'' abbiano più importanza di altri reati commessi nel calcolo della posizione semantica del criminale stesso.